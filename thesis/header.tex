\usepackage[utf8]{inputenc}
\usepackage[T1]{fontenc}
\usepackage{amsmath}
\usepackage{amsthm}
\usepackage{amssymb}
%\usepackage{rotating}
%\usepackage{amslatex}
\usepackage{siunitx}
\usepackage{multicol}%for multicol
\usepackage{blkarray}%blockarray and block
\usepackage{comment}
\usepackage{fnbreak}%get a warning if a footnote is split
\usepackage[section]{placeins}
\usepackage{listings}
\lstset{breaklines=true,basicstyle=\ttfamily,language=Python}
\usepackage{arrayjobx}
\usepackage{array}%for \newcolumntype
%\usepackage[shortlabels]{enumitem}
\usepackage{mathtools}
\usepackage{afterpage}
\usepackage{setspace}% for \setstretch
\usepackage{algorithm}
\usepackage{algpseudocode}%for algorithmic
\usepackage{thmtools}%so that autoref works with Lemmas
\usepackage{tikz}
\usepackage{pgfplots}
\pgfplotsset{compat=1.15}
\usepackage{shuffle}
\usepackage{textcomp}%for \textrecipe
\usepackage{fontawesome}%for \faTable
\usetikzlibrary{calc,shapes,arrows.meta,decorations.markings,arrows}
\usetikzlibrary{graphs,positioning,svg.path,backgrounds}
\newcommand{\tikzmark}[1]{\tikz[overlay,remember picture] \node (#1) {};}
%\usepackage{CJKutf8}%for CJKChar

%\usepackage[backend=biber,backref=true]{biblatex}
\usepackage[backend=biber,style=alphabetic,backref=true,maxbibnames=10]{biblatex}
\addbibresource{sigs.bib}
\usepackage{url}

\usepackage{imakeidx}
%not a list of definitions, just symbols and abbreviations
%What is an abbreviation? Is QR? 
\makeindex[intoc,title=Symbols and abbreviations index]
\def\jind#1{\index{#1}}
\def\jindmath#1#2{\index{#2@$#1$}}
\def\jindv#1{\index{#1v@\texttt{#1}}}
%Some places I've given up and used \index in the text 
\definecolor{bluee}{rgb}{0.4, 0.4, 1.0}
%https://tex.stackexchange.com/questions/134191/line-breaks-of-long-urls-in-biblatex-bibliography
\setcounter{biburlucpenalty}{8000}
\setcounter{biburllcpenalty}{8000}

\usepackage{hyperref}

%so that autoref works with algorithms
\newcommand{\algorithmautorefname}{Algorithm}

%might help url breaking in bibliography
%\Urlmuskip=0mu plus 1mu minus 1mu
%also all the emergencystretch/looseness/fussy/sloppy
%to play with
%https://tex.stackexchange.com/questions/18505/how-to-use-sloppy-for-just-some-references

\def\ii{{\texttt{iisignature}}}
\def\pypi{{\texttt{PyPI}}}
\def\numpy{{\texttt{numpy}}}
\def\scipy{{\texttt{scipy}}}
\def\i#1{\index{#1@\texttt{#1}}}
\def \hilite#1{\underline{\color{blue}\textbf{#1}}}
%\def \alph#1{{\color{blue}\mathbf{#1}}}
\def \lex{<_L}
\def\kron{\underline{\otimes}}

\graphicspath{{C:/Users/Jeremy/Dropbox/phd/graphs/}{/home/jeremyr/Dropbox/phd/graphs/}{/Users/reizenstein/DropboxPersonalSymlink/phd/graphs/}}

%\RequirePackage{relsize}
%\DeclareRobustCommand\CXX{C\kern-.05em \raisebox{.3ex}{\scalebox{0.9}{\textbf{+\kern-.10em+}}}}
\DeclareRobustCommand\CXX{C\kern-.05em {\scalebox{0.9}{\textbf{+\kern-.10em+}}}}
%\DeclareRobustCommand\{\texorpdfstring{\CXX}{C++}}
\DeclareRobustCommand\CC{C\texttt{++}}
\def\bftab{\fontseries{b}\selectfont}
\newtheorem{theorem}{Theorem}
%\newtheorem*{theorem*}{Theorem}%bad idea, because you can't refer to it.
\newtheorem{definition}[theorem]{Definition}
%\newtheorem{outsideTheorem}[theorem]{Theorem}
\newtheorem{example}[theorem]{Example}
\newtheorem{conjecture}[theorem]{Conjecture}
\newtheorem{lemma}[theorem]{Lemma}
\newtheorem{proposition}[theorem]{Proposition}
\newtheorem{remark}[theorem]{Remark}

\newcommand{\area}{\mathsf{area}}
\newcommand{\Area}{\mathsf{Area}}
\newcommand{\emptyword}{\epsilon}
\newcommand{\ds}{d} % dimension of the signal
\newcommand{\TC}{T((\R^\ds))} % concat
\newcommand{\TS}{T(\R^\ds)} % shuffle
\newcommand{\GL}{\operatorname{GL}}
\newcommand{\SO}{\operatorname{SO}}
%\newcommand{\id}{\operatorname{id}}
\newcommand{\id}{\mathsf{id}}

\newcommand{\evaluatedAt}[1]{\,\raisebox{-.5em}{$\vert_{#1}$}}

\def\hssymbol{\mathbin{\succ}}
\def\hs#1#2{#1\hssymbol#2} %half shuffle
%\def\hs#1#2{z(#1,#2)} %half shuffle
\def\areab#1{\underline{\area}(#1)}
\def\areabb{\underline{\area}}
\newcommand{\R}{\mathbb{R}}
\newcommand{\Q}{\mathbb{Q}}
\newcommand{\C}{\mathbb{C}}
\newcommand{\N}{\mathbb{N}}
\newcommand{\spann}{\operatorname{span}}
\newcommand{\sign}{\operatorname{sign}}

\DeclareMathOperator*{\argmax}{arg\,max}
\DeclareMathOperator*{\argmin}{arg\,min}
\DeclareMathOperator{\softmax}{softmax}

%indicate that this file has been had
\def\headerIncludedJR{}
\def\endDocumentJR{}

%general hints
%https://homepages.inf.ed.ac.uk/imurray2/compnotes/latex.html
